\documentclass{tufte-handout}
\usepackage{amsmath,amsthm}

\usepackage{booktabs}
\usepackage{graphicx}
\usepackage[separate-uncertainty]{siunitx}
\usepackage{tikz}

\newtheorem{claim}{Claim}[section]
\title{\sf Marking Trees}
\date{}
\begin{document}
\maketitle

\section{Lab Report: Marking Trees}


by Marcus de Lacerda and Martin Larsson\sidenote{Complete the report by filling
  in your names and the parts marked $[\ldots]$.
  Remove the sidenotes in your final hand-in.}

\subsection{Results}

For $i\in\{1,2,3\}$, the number of rounds $R_i$ spent until the tree
is completely marked in process $i$ is given in the following table.
The table shows the result of 10 repeated
trails.\footnote{Report your empirical data.
  Give each value as the mean plus/minus one standard deviation.
  Use whatever best practices for reporting data you may have learned;
  here's a crash course that suffices for our purposes: (i) Calculate mean and standard deviation ($m
  = 2.5074$, $s = 0.889341021813$) from a number of repeated
  experiments.
  (ii) Round $s$ to one significant digit ($s = 0.9$).
  (iii) Round $m$ to the decimal place corresponding to the first
  significant digit in $s$ ($m = 2.5$, $s = 0.9$).
  (iv) Report $m\pm s$ ($2.5 \pm 0.9$).
  (v) Use scientific notation.

  If you've taken too many statistics classes, feel free to go to town
  with graphs and confidence intervals and so forth.}

\medskip\noindent
\begin{tabular}{
    S[table-format = 7]
    S[table-format = 1.1(1)e1]
    S[table-format = 1.1(1)e1]
    S[table-format = 1.1(1)e1]
  }
% WARNING: This table the (brilliant) siunitx package.
% This allows typesetting of nicely aligned numbers.
% If this is too much to absorb, just use a normal Latex table.
% (Or do the table in another tool, export as PDF, and include it.)
% Or do the whole report in your favourite word processor instead.
\toprule
{ $N$ } & { $R_1$ } & {$R_2$} & {$R_3$} \\\midrule
3 & 2.6\pm 0.9 \\
7 & \\
15 & \\
31 & \\
63 & \\
127 & \\
255 & \\
511 & \\
1023 & 3.2 \pm 0.5 e3\\
$\vdots$ \\
524287 & 3.2 \pm 0.2 e6 \\
1048575 \\\bottomrule
\end{tabular}

\subsection{Analysis}

Our experimental data indicates that $\mathbf E [R_1]$ is [$\ldots$],
while $\mathbf E[R_2]$ [$\cdots$], and $\mathbf E[R_3]$
[$\cdots$].\footnote{For each of the tree processes, try to express
  the observed behaviour of $R_i$ using standard terminology from the
  analys of algorithms.
  For instance, use expressions such as ``$E[R_1]$ is logarithmic in
  $N$'' or ``$E[R_2]$ is somewhere between $\Omega(N^{1/2})$ and
  $\Omega(N^{3/2})$''.}

Theoretically, the behaviour of $R_1$ can be explained as follows: [$\cdots$] \footnote{This
  is the difficult part.
  You need to write a few lines that explain the random process
  underlying $R_1$ and derive an expression for $\mathbf E[R_1]$.
  (Hopefully it's the same as your empirical analysis!)
  Once you recognize what's going on, this should be easy; it involves
  no complicated calculations.

  \emph{Hint:} If you're stuck at $R_1$, do the following experiment
  as a warmup.
  Process $0$ is like process $1$, except that Bob doesn't use his
  marking rules: the only nodes that get marked are those sent by
  Alice. Implement this (it's easy) and analyse the behaviour both
  theoretically and practically.

  \emph{Optional:} Explain the
  behaviours of $R_2$ and $R_3$ as well.
  The behaviour of $R_2$ is quite a bit harder; while $R_3$ is just
  cute.
}



\newpage
\section{Perspective}

This lab establishes minimal skills in simulation of random processes,
introduces the Knuth shuffle for those who haven't seen it, and some
basic probability theory about occupancy problems.
In process 2, Alice's messages are not independent, which can lead to
temping errors in the analysis.
Process 3 is just surprising (and maybe fun to implement), but not
much about randomness is to be learned from it.

The exercise is built on an assigment of Michael
Mitzenmacher.\sidenote{Michael Mitzenmacher, An Experimental
  Assignment on Random Processes, SIGACT News, 27 December 2000.
  See also section 5.8 in M.
  Mitzenmacher and E.
  Upfal, Probability and Computing -- Randomized Algorithms and
  Probabilistic Analysis.
  Cambridge University Press, 2005.}

The front page image shows \emph{Acinonyx jubatus} marking a
tree. Photo by Joachim Huber, under the Creative Commons Attribution-Share
Alike 2.0 Generic license.\sidenote{commons.wikimedia.org/\-wiki/\-File:\-Acinonyx\_jubatus\_-\-Southern\_Namibia-8.jpg}

\end{document}
